\documentclass[11pt]{exam}
\newcommand{\myname}{Sihao Yin, Yuxuan Jiang} %Write your name in here

\newcommand{\myUCO}{0028234022, 0028440468} %write your UCO in here

\newcommand{\myhwtype}{Homework}
\newcommand{\myhwnum}{17} %Homework set number
\newcommand{\myclass}{CS580}
\newcommand{\mylecture}{}
\newcommand{\mysection}{}
\usepackage{listings}
% Prefix for numedquestion's
\newcommand{\questiontype}{Question}

% Use this if your "written" questions are all under one section
% For example, if the homework handout has Section 5: Written Questions
% and all questions are 5.1, 5.2, 5.3, etc. set this to 5
% Use for 0 no prefix. Redefine as needed per-question.
\newcommand{\writtensection}{0}

\usepackage{amsmath, amsfonts, amsthm, amssymb}  % Some math symbols
\usepackage{enumerate}

\usepackage{graphicx}
\usepackage{hyperref}
\usepackage[all]{xy}
\usepackage{wrapfig}
\usepackage{fancyvrb}
\usepackage[T1]{fontenc}
\usepackage{listings}
\usepackage[shortlabels]{enumitem}

\usepackage{centernot}
\usepackage{mathtools}
\DeclarePairedDelimiter{\ceil}{\lceil}{\rceil}
\DeclarePairedDelimiter{\floor}{\lfloor}{\rfloor}
\DeclarePairedDelimiter{\card}{\vert}{\vert}


\setlength{\parindent}{0pt}
\setlength{\parskip}{5pt plus 1pt}
\pagestyle{empty}

\def\indented#1{\list{}{}\item[]}
\let\indented=\endlist

\newcounter{questionCounter}
\newcounter{partCounter}[questionCounter]

\newenvironment{namedquestion}[1][\arabic{questionCounter}]{%
    \addtocounter{questionCounter}{1}%
    \setcounter{partCounter}{0}%
    \vspace{.2in}%
        \noindent{\bf #1}%
    \vspace{0.3em} \hrule \vspace{.1in}%
}{}

\newenvironment{numedquestion}[0]{%
	\stepcounter{questionCounter}%
    \vspace{.2in}%
        \ifx\writtensection\undefined
        \noindent{\bf \questiontype \; \arabic{questionCounter}. }%
        \else
          \if\writtensection0
          \noindent{\bf \questiontype \; \arabic{questionCounter}. }%
          \else
          \noindent{\bf \questiontype \; \writtensection.\arabic{questionCounter} }%
        \fi
    \vspace{0.3em} \hrule \vspace{.1in}%
}{}

\newenvironment{alphaparts}[0]{%
  \begin{enumerate}[label=\textbf{(\alph*)}]
}{\end{enumerate}}

\newenvironment{arabicparts}[0]{%
  \begin{enumerate}[label=\textbf{\arabic{questionCounter}.\arabic*})]
}{\end{enumerate}}

\newenvironment{questionpart}[0]{%
  \item
}{}

\newcommand{\answerbox}[1]{
\begin{framed}
\vspace{#1}
\end{framed}}

\pagestyle{head}

\headrule
\header{\textbf{\myclass\ \mylecture\mysection}}%
{\textbf{\myname\ (\myUCO)}}%
{\textbf{\myhwtype\ \myhwnum}}

\begin{document}
\thispagestyle{plain}
\begin{center}
  {\Large \myclass{} \myhwtype{} \myhwnum} \\
  \myname{} (\myUCO{}) \\
  \today
\end{center}


%Here you can enter answers to homework questions

\begin{numedquestion}
We can prove the inequality using derivatives. Suppose f(x) = $e^x - 1 - x$, then f'(x) = $e^x - 1$. f'(x) = 0 when x = 0 and f'(x) > 0 when x > 0.Hence f(x) is always increasing when x > 0. Since f(0) = 0, we know $e^x \ge 1 + x$

\end{numedquestion}

\pagebreak
\begin{numedquestion}
If there is a cost, then we can redefine OPT to be the total cost of the optimal set cover. 

The first question would also be changed to how big is the cost of F, c(F), relative to OPT. Hence, E(c(F)) = $\sum_j Pr[S_j \in F] * c_j$ $\le$ OPT. The proof for this part is the same as in the lecture.  

Hence, when adding a $\alpha$ as in the lecture to do randomized rounding, we have E(c(F)) = $\sum_j \alpha * Pr[S_j \in F] * c_j$ $\le \alpha$  OPT. Since the cost is non-negative, we can use Markov's inequality and we have Pr(c(F) $\ge$ ((1+$\frac{2}{ln(m)})\alpha$OPT)) $\le$ $\frac{1}{1 + \frac{2}{ln(m)}}$ $\le$ 1 - $\frac{2}{ln(m)}$ $\le$ 1 - $\frac{1}{ln(m)}$

Now, let's see the second question, is F a set cover? We perform the same analysis as in the lecture. Pr(F not a set cover) = Pr(F does not cover some i) $\le$ $\sum_{i=1}^m$ Pr(F does not cover i) $\le$ m $e^{-\alpha}$

If we take $\alpha$ = 2ln(m), then for the first question, $(1+\frac{2}{ln(m)})\alpha$ = 2ln(m) + 4. Hence, Pr(c(F) $\le$ (2ln(m) + 4)OPT)) $\ge$ $\frac{1}{ln(m)}$. For the second question, we have Pr(F not a set cover) $\le$ $\frac{1}{m}$ 

Above is the O(ln(m)) approximation
\end{numedquestion}



% if you do not solve some of the questions use this command to increment counter
%\setcounter{questionCounter}{4}
%\begin{numedquestion}
%  Questions 2 and 3 were not solved, this is an answer to question 5.
%\end{numedquestion}


% if questions have subparts, use this command
%\pagebreak
%\begin{numedquestion}
%  Use the alphaparts environment to for letters.
%  \begin{alphaparts}
%    \item Part a
%    \item Part b
%    \item Part c
%  \end{alphaparts}
%\end{numedquestion}


%\begin{numedquestion}
%  Using the \texttt{description} environment is a great way to typeset induction proofs!
%  \begin{description}
%    \item[Base Case:]
%      Here I have my base case.
%    \item[Induction Hypothesis:]
%      Assume things to make proof work. 
%    \item[Induction Step:]
%      Prove all the things.
%  \end{description}

%  Therefore, we have proven the claim by induction on in the \texttt{description} environment.
%\end{numedquestion}



\end{document}
