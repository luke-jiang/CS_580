\documentclass[11pt]{exam}
\newcommand{\myname}{Sihao Yin, Yuxuan Jiang} %Write your name in here

\newcommand{\myUCO}{0028234022, 0028440468} %write your UCO in here

\newcommand{\myhwtype}{Homework}
\newcommand{\myhwnum}{5} %Homework set number
\newcommand{\myclass}{CS580}
\newcommand{\mylecture}{}
\newcommand{\mysection}{}
\usepackage{listings}
% Prefix for numedquestion's
\newcommand{\questiontype}{Question}

% Use this if your "written" questions are all under one section
% For example, if the homework handout has Section 5: Written Questions
% and all questions are 5.1, 5.2, 5.3, etc. set this to 5
% Use for 0 no prefix. Redefine as needed per-question.
\newcommand{\writtensection}{0}

\usepackage{amsmath, amsfonts, amsthm, amssymb}  % Some math symbols
\usepackage{enumerate}
\usepackage{enumitem}
\usepackage{graphicx}
\usepackage{hyperref}
\usepackage[all]{xy}
\usepackage{wrapfig}
\usepackage{fancyvrb}
\usepackage[T1]{fontenc}
\usepackage{listings}

\usepackage{centernot}
\usepackage{mathtools}
\DeclarePairedDelimiter{\ceil}{\lceil}{\rceil}
\DeclarePairedDelimiter{\floor}{\lfloor}{\rfloor}
\DeclarePairedDelimiter{\card}{\vert}{\vert}


\setlength{\parindent}{0pt}
\setlength{\parskip}{5pt plus 1pt}
\pagestyle{empty}

\def\indented#1{\list{}{}\item[]}
\let\indented=\endlist

\newcounter{questionCounter}
\newcounter{partCounter}[questionCounter]

\newenvironment{namedquestion}[1][\arabic{questionCounter}]{%
    \addtocounter{questionCounter}{1}%
    \setcounter{partCounter}{0}%
    \vspace{.2in}%
        \noindent{\bf #1}%
    \vspace{0.3em} \hrule \vspace{.1in}%
}{}

\newenvironment{numedquestion}[0]{%
	\stepcounter{questionCounter}%
    \vspace{.2in}%
        \ifx\writtensection\undefined
        \noindent{\bf \questiontype \; \arabic{questionCounter}. }%
        \else
          \if\writtensection0
          \noindent{\bf \questiontype \; \arabic{questionCounter}. }%
          \else
          \noindent{\bf \questiontype \; \writtensection.\arabic{questionCounter} }%
        \fi
    \vspace{0.3em} \hrule \vspace{.1in}%
}{}

\newenvironment{alphaparts}[0]{%
  \begin{enumerate}[label=\textbf{(\alph*)}]
}{\end{enumerate}}

\newenvironment{arabicparts}[0]{%
  \begin{enumerate}[label=\textbf{\arabic{questionCounter}.\arabic*})]
}{\end{enumerate}}

\newenvironment{questionpart}[0]{%
  \item
}{}

\newcommand{\answerbox}[1]{
\begin{framed}
\vspace{#1}
\end{framed}}

\pagestyle{head}

\headrule
\header{\textbf{\myclass\ \mylecture\mysection}}%
{\textbf{\myname\ (\myUCO)}}%
{\textbf{\myhwtype\ \myhwnum}}

\begin{document}
\thispagestyle{plain}
\begin{center}
  {\Large \myclass{} \myhwtype{} \myhwnum} \\
  \myname{} (\myUCO{}) \\
  \today
\end{center}


%Here you can enter answers to homework questions

\begin{numedquestion}
  First note if the car keeping going horizontally for t steps, it will end up with velocity (t,0). Since we start at position (1,y), the car will end up at column $\sum_1^t i = \frac{t(t+1)}{2}$. Since the car cannot go out out of the track, $\frac{t(t+1)}{2} \leq n$, hence $t \leq \sqrt{2n}$. Therefore, the velocity of the car, both horizontally and vertically, cannot exceed $\sqrt{2n}$.
  
  Now we construct the graph. We argue that we can construct a directed graph G with O($n^3$) vertices and O($n^3$) edges:

  \begin{enumerate}
      \item vertex: a vertex in G can be represented as (x,y,$v_x$,$v_y$), with $v_x$ and $v_y$ representing the horizontal and vertical velocities respectively. Since each (x,y) is valid if it is inside the track, we know $1 \leq x \leq n$,$1 \leq y \leq n$. Also, from our reasoning above, we know velocity cannot exceed $\sqrt{2n}$, hence $-\sqrt{2n} \leq v_x \leq \sqrt{2n}$, $-\sqrt{2n} \leq v_y \leq \sqrt{2n}$. Therefore, there are O(n*n*$\sqrt{n}$*$\sqrt{n}$) = O($n^3$) vertices 
      
      \item edges: There is a directed edge between two vertices (x,y,$v_x$,$v_y$) and (x',y',$v_x'$,$v_y'$) if both (x,y) and (x',y') are inside the track and the following conditions are satisfied 
      \begin{enumerate}
          \item x' = x + $v_x$ and $v_x' \in \{ v_x -1 , v_x, v_x + 1\}$
          \item y' = y + $v_y$ and $v_y' \in \{ v_y -1, v_y, v_y + 1\}$
      \end{enumerate}
       Since each edge represent a legal move from one vertex to another, and there are constant number of moves for each vertex, we have O($n^3$) edges in G
  \end{enumerate}
  
  We can also create a dummy source and sink. The dummy source has O(n) outgoing edges to vertices in the starting area, since vertices in the starting area has the form (1,y,0,0). The dummy sink has O($n^2$) incoming edges from the finish area, since vertices in the finish area has the form (n,y,$v_x$,$v_y$).
  
  We can use BFS to find the shortest path between the dummy source and sink in order to find the minimum steps we want, since the minimum steps = length of shortest path between source and sink - 2. We also need an auxiliary array to store the distance from a vertex v to the source s. 
  
  Below is the pesudocode assuming there are V vertices and E edges.   
  \begin{lstlisting}
    dist = an array of size V storing the distance between dist[i] to source
    
    function minSteps(G):
        q = a queue of vertices to visit 
        
        insert s to q
        mark s as visited 
        dist[s] = 0
        
        while q is not empty:
            v = pop q
            for all not visited neighbours v' of v:
                mark v' as visited 
                dist[v'] = dist[v] + 1
                push v' to the end of the q 
                
                if v' is sink vertex:
                    return dist[sink]
                
  \end{lstlisting}
  
  Analysis:
    \begin{enumerate}
        \item The above algorithm is correct because the earlier a vertex is visited, the smaller its value in the array dist will be, since the graph is not weighted. 
        \item Since the above algorithm is BFS, the time complexity will be O(V+E) = O($n^3$) 
    \end{enumerate}
\end{numedquestion}

\pagebreak
\begin{numedquestion}
   We can use Dijkstra's algorithm to solve this problem. We first use DFS to find negative cycles. If there is a negative cycle reachable from s, report it and terminate the algorithm. If there isn't, we first remove the two negative edges, then we run Dijkstra's algorithm three times. First without accounting any of the negative edges, then accounting just one edge, then accounting just the last edge. We also keep updating the shortest distance from s. We compare the distances between accounting the negative edge and without accounting it, and find the minimal of these two.  
   
   Below is the pseudocode 
    \begin{lstlisting}
                    
    function shortestPath(G, s):
        E1(v1, w1), E2(v2, w2) are two negative edges
        run DFS(w1) and label all vertices reachable from w1
        run DFS(w2) and label all vertices reachable from w2
        if (vi = {v1, v2} is labelled):
            pathsum = sum of weights from wi to vi
            if (pathsum + weight(Ei) < 0):
                report negative cycle
        
        delete the two negative edges E1(u1, v1), E2(u2, v2) from G
        
        run Dijkstra(s) with label dist
        
        run Dijkstra(v1) with label dist1
        
        for each vertex v, update its true distance as
            v.dist = min(v.dist, v.dist1 + u1.dist + weight(E1))
            
        run Dijkstra(v2) with label dist2
        for each vertex v, update its true distance as
            //this is the final shortest distance for each v
            v.dist = min(v.dist, v.dist2 + u2.dist + weight(E2))
            
            
    \end{lstlisting}
  
  
\textbf{Analysis:} 
\begin{enumerate}
    \item The above algorithm is correct because, if there is a negative cycle containing an negative edge E(u, v), then by definition u is reachable from v, so we can use a DFS to detect cycles. Once a cycle is detected, we can see if it is a negative cycle by summing up the weights. When we have detected a negative cycle, the algorithm stops and reports the result. Otherwise, we are sure that the existence of a negative edge will not endlessly cause the weight of certain vertices to decrease. In this case, we can remove the two negative edges and run Dijkstra to find out the shortest paths from s that don't go through negative edges. Then, consider each negative edge iteratively. Adding a negative edge to the graph would potentially causing distances of certain vertices to decrease, so for each vertex, the true distance is the minimum of its distance without a negative edge and the shortest path that goes through the negative edge, which equals to the distance from s to u, plus the weight of edge(u, v), plus the distance from v to that vertex. So we need another Dijkstra run to find out the shortest path from v to each vertex. After updating, the distance label would be the true distance for each vertex, and we repeat the same procedure for the other negative edge.
    
    \item The algorithm first runs DFS two times to detect negative cycle, which takes O(m + n) time. After that, calculating pathsum takes O(m) time for each cycle case. After that, each Dijkstra takes O(m + nlogn) time and each distance update takes O(n) time. So the overall runtime is O(m + n) + O(m) + O(m + nlogn) + O(n) = O(m + nlogn).
    
    
\end{enumerate}
      
\end{numedquestion}

\pagebreak
\begin{numedquestion}
   Since we are given a dag, we can do a topological sort to help us. Below is the pseudocode 
   
   \begin{lstlisting}
        function shortest(G):
            stack = a stack storing the topological order of all vertices 
            dist = an array storing the distance from each vertex to source
            
            dist[s] = 0
            for all vertices v other than s:
                dist[v] = Infinity
            
            while stack is not empty:
                v = pop stack
                for all adjacent vertex v' of v:
                    if dist[v'] > dist[v] + weight of the edge from v to v':
                        dist[v'] = dist[v] + weight of the edge from v to v'
            
   \end{lstlisting}
   
   Analysis: First, we performed a topological sort of this dag, hence the time complexity for that is O(V+E). Once we have the topological order, we traversed all the vertices, the time complexity for this is O(V). Then for each vertex, we calculated the distance using the weight of the edge, the time complexity for this is O(E), since there are O(E) pair of adjacent vertices. Hence, in total, the time complexity is O(V+E).  
\end{numedquestion}

% if you do not solve some of the questions use this command to increment counter
%\setcounter{questionCounter}{4}
%\begin{numedquestion}
%  Questions 2 and 3 were not solved, this is an answer to question 5.
%\end{numedquestion}


% if questions have subparts, use this command
%\pagebreak
%\begin{numedquestion}
%  Use the alphaparts environment to for letters.
%  \begin{alphaparts}
%    \item Part a
%    \item Part b
%    \item Part c
%  \end{alphaparts}
%\end{numedquestion}


%\begin{numedquestion}
%  Using the \texttt{description} environment is a great way to typeset induction proofs!
%  \begin{description}
%    \item[Base Case:]
%      Here I have my base case.
%    \item[Induction Hypothesis:]
%      Assume things to make proof work. 
%    \item[Induction Step:]
%      Prove all the things.
%  \end{description}

%  Therefore, we have proven the claim by induction on in the \texttt{description} environment.
%\end{numedquestion}



\end{document}
