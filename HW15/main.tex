\documentclass[11pt]{exam}
\newcommand{\myname}{Sihao Yin, Yuxuan Jiang} %Write your name in here

\newcommand{\myUCO}{0028234022, 0028440468} %write your UCO in here

\newcommand{\myhwtype}{Homework}
\newcommand{\myhwnum}{14} %Homework set number
\newcommand{\myclass}{CS580}
\newcommand{\mylecture}{}
\newcommand{\mysection}{}
\usepackage{listings}
% Prefix for numedquestion's
\newcommand{\questiontype}{Question}

% Use this if your "written" questions are all under one section
% For example, if the homework handout has Section 5: Written Questions
% and all questions are 5.1, 5.2, 5.3, etc. set this to 5
% Use for 0 no prefix. Redefine as needed per-question.
\newcommand{\writtensection}{0}

\usepackage{amsmath, amsfonts, amsthm, amssymb}  % Some math symbols
\usepackage{enumerate}

\usepackage{graphicx}
\usepackage{hyperref}
\usepackage[all]{xy}
\usepackage{wrapfig}
\usepackage{fancyvrb}
\usepackage[T1]{fontenc}
\usepackage{listings}
\usepackage[shortlabels]{enumitem}

\usepackage{centernot}
\usepackage{mathtools}
\DeclarePairedDelimiter{\ceil}{\lceil}{\rceil}
\DeclarePairedDelimiter{\floor}{\lfloor}{\rfloor}
\DeclarePairedDelimiter{\card}{\vert}{\vert}


\setlength{\parindent}{0pt}
\setlength{\parskip}{5pt plus 1pt}
\pagestyle{empty}

\def\indented#1{\list{}{}\item[]}
\let\indented=\endlist

\newcounter{questionCounter}
\newcounter{partCounter}[questionCounter]

\newenvironment{namedquestion}[1][\arabic{questionCounter}]{%
    \addtocounter{questionCounter}{1}%
    \setcounter{partCounter}{0}%
    \vspace{.2in}%
        \noindent{\bf #1}%
    \vspace{0.3em} \hrule \vspace{.1in}%
}{}

\newenvironment{numedquestion}[0]{%
	\stepcounter{questionCounter}%
    \vspace{.2in}%
        \ifx\writtensection\undefined
        \noindent{\bf \questiontype \; \arabic{questionCounter}. }%
        \else
          \if\writtensection0
          \noindent{\bf \questiontype \; \arabic{questionCounter}. }%
          \else
          \noindent{\bf \questiontype \; \writtensection.\arabic{questionCounter} }%
        \fi
    \vspace{0.3em} \hrule \vspace{.1in}%
}{}

\newenvironment{alphaparts}[0]{%
  \begin{enumerate}[label=\textbf{(\alph*)}]
}{\end{enumerate}}

\newenvironment{arabicparts}[0]{%
  \begin{enumerate}[label=\textbf{\arabic{questionCounter}.\arabic*})]
}{\end{enumerate}}

\newenvironment{questionpart}[0]{%
  \item
}{}

\newcommand{\answerbox}[1]{
\begin{framed}
\vspace{#1}
\end{framed}}

\pagestyle{head}

\headrule
\header{\textbf{\myclass\ \mylecture\mysection}}%
{\textbf{\myname\ (\myUCO)}}%
{\textbf{\myhwtype\ \myhwnum}}

\begin{document}
\thispagestyle{plain}
\begin{center}
  {\Large \myclass{} \myhwtype{} \myhwnum} \\
  \myname{} (\myUCO{}) \\
  \today
\end{center}


%Here you can enter answers to homework questions

\begin{numedquestion}
No. According to the proof, we only utilized pair-wise independence 
\end{numedquestion}

\pagebreak
\begin{numedquestion}
We define the random variables Xi to be the balls a particular bin has. Xi = 1 means the ball Xi is in the bin, Xi = 0 means otherwise. Hence, suppose max load is k, we want P($\sum_{i=1}^{n}Xi$ $\ge$ k) $\le \delta$\\  

Using theorem 1 from the lecture note, we have P($\sum_{i=1}^{n}Xi$ $\ge$ k) = $\frac{pn+3{(pn)}^2}{{(k-pn)}^4}$ $\le \delta$\\

Step1: We raise both side to the power of $\frac{1}{4}$, we get $\frac{{(pn+3{(pn)}^2)}^\frac{1}{4}}{{(k-pn)}}$ $\le {\delta}^\frac{1}{4}$\\

Step2: With the above, we have $\frac{k-pn}{{(pn+3{(pn)}^2)}^\frac{1}{4}}$ $\ge$ $\frac{1}{{\delta}^\frac{1}{4}}$ \\

Step3: Simplify above, we have $\frac{k}{{(pn)}^\frac{1}{4}}$ $\ge$ $\frac{k-pn}{{(pn+3{(pn)}^2)}^\frac{1}{4}}$ $\ge$ $\frac{1}{{\delta}^\frac{1}{4}}$ \\

Step4: With the above, we have k $\ge$ $\frac{{(pn)}^\frac{1}{4}}{{\delta}^\frac{1}{4}}$\\

Step5: Since $p^\frac{1}{4}$ is a constant, we have k $\le$ O($\frac{{n}^\frac{1}{4}}{{\delta}^\frac{1}{4}}$) with the probability of 1 - $\delta$
\end{numedquestion}

\pagebreak
\begin{numedquestion}
Suppose P(Xn = xn) = p, then P(Xn $\ne$ xn) = (1-p).\\
Suppose ci is a value $\in$ [x1,xn] that is drawn at the ith time. Note the value of ci is random.\\
Due to K-wise independence, we have P(X1 = c1, X2 = c2, ...., Xn = xn) = P(X1=c1)P(X2=c2)...p. Similarly, P(X1 = c1, X2 = c2, ...., Xn $\ne$ xn) = P(X1=c1)P(X2=c2)...(1-p).\\

Hence, P(X1 = c1, X2 = c2, ....,Xn-1 = cn-1) = P(X1=c1)P(X2=c2)...p + P(X1=c1)P(X2=c2)...(1-p) = P(X1=c1)P(X2=c2)..P(Xn-1=cn-1), which is (k-1)-wise independence
\end{numedquestion}

% if you do not solve some of the questions use this command to increment counter
%\setcounter{questionCounter}{4}
%\begin{numedquestion}
%  Questions 2 and 3 were not solved, this is an answer to question 5.
%\end{numedquestion}


% if questions have subparts, use this command
%\pagebreak
%\begin{numedquestion}
%  Use the alphaparts environment to for letters.
%  \begin{alphaparts}
%    \item Part a
%    \item Part b
%    \item Part c
%  \end{alphaparts}
%\end{numedquestion}


%\begin{numedquestion}
%  Using the \texttt{description} environment is a great way to typeset induction proofs!
%  \begin{description}
%    \item[Base Case:]
%      Here I have my base case.
%    \item[Induction Hypothesis:]
%      Assume things to make proof work. 
%    \item[Induction Step:]
%      Prove all the things.
%  \end{description}

%  Therefore, we have proven the claim by induction on in the \texttt{description} environment.
%\end{numedquestion}



\end{document}
