\documentclass{article}
\usepackage[utf8]{inputenc}
\usepackage[colorlinks]{hyperref}
\usepackage{amsmath}

\newcommand{\reals}{\mathbb{R}}


\title{Midterm 2 Practice Problems}
\author{CS580}

\begin{document}

\maketitle

Here is a running list of practice problems for the second midterm. Midterm 2 is on \textbf{Tuesday, March 10} at 8:00 PM in MATH 175. Feel free to run your solutions by any of the staff. You are strongly encouraged to discuss the problems with classmates on Piazza or elsewhere.

\begin{enumerate}
    \item Any problem in \href{http://jeffe.cs.illinois.edu/teaching/algorithms/book/03-dynprog.pdf}{Chapter 3} of Jeff's notes. Favorites include 2, 3, 4, 5, 9, 25, 33, 42,
    \item Any problem in Chapter 6 of Kleinberg-Tardos. Favorites include 5, 6, 11, 13, 17, 23, 25.
    \item Any problem in
    \href{http://jeffe.cs.illinois.edu/teaching/algorithms/book/07-mst.pdf}{Chapter 7} of Jeff's notes. Favorites include 5, 6, 9.
    \item Any problem in Chapter 7 of Kleinberg-Tardos. Favorites include 7, 8, 12, 20, 22, 24, 36, 50, as well as the several applications in the text.
    \item Any problem in \href{http://jeffe.cs.illinois.edu/teaching/algorithms/book/10-maxflow.pdf}{Chapter 10} of Jeff's notes. Favorites include 5, 7, 12, 14
    \item Any problem in
    \href{http://jeffe.cs.illinois.edu/teaching/algorithms/book/11-maxflowapps.pdf}{Chapter 11} of Jeff's notes. First, the examples in this chapter are very good and can be treated as practice problems with solutions provided. After that, favorite exercises include 4, 8, 9, 15, 16, 19.
    \item Any problem in \href{http://jeffe.cs.illinois.edu/teaching/algorithms/notes/G-mincostflow.pdf}{Chapter G} of Jeff's notes (to somewhat lesser extent than the other flow chapters).  We are particularly interested in those problems that emphasize \emph{applications} of min cost flow. Favorites include 3, 4, 5. 
    \item From \href{http://jeffe.cs.illinois.edu/teaching/algorithms/book/12-nphard.pdf}{Chapter 12} of Jeff's notes: first, it would be good to go over the other examples of reductions from SAT for some of the canonical problems, and be familiar with the problems listed in section 13, which are known to be as hard as SAT and would be useful to reduce from. After that, favorite exercises include 1, 4, 5, 6, 7, 10, 12, 16, 17, 23, 25, 26, 27, 28, 34, 39.
    \item Any problem in Chapter 8 of Kleinberg-Tardos. It would also be good to go over their examples of NP-Hard problems for some of the standard hard problems. Favorites include 2, 3, 5, 6, 7, 12.
    \item Any problem in \href{https://people.eecs.berkeley.edu/~vazirani/algorithms/chap8.pdf}{Chapter 8} of Dasgupta-Papadimitriou-Vazirani. The examples in the text are also useful and can be as exercises with provided solutions.
\end{enumerate}


\end{document}
