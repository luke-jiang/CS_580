\documentclass[11pt]{exam}
\newcommand{\myname}{Sihao Yin, Yuxuan Jiang} %Write your name in here

\newcommand{\myUCO}{0028234022, 0028440468} %write your UCO in here

\newcommand{\myhwtype}{Homework}
\newcommand{\myhwnum}{9} %Homework set number
\newcommand{\myclass}{CS580}
\newcommand{\mylecture}{}
\newcommand{\mysection}{}
\usepackage{listings}
% Prefix for numedquestion's
\newcommand{\questiontype}{Question}

% Use this if your "written" questions are all under one section
% For example, if the homework handout has Section 5: Written Questions
% and all questions are 5.1, 5.2, 5.3, etc. set this to 5
% Use for 0 no prefix. Redefine as needed per-question.
\newcommand{\writtensection}{0}

\usepackage{amsmath, amsfonts, amsthm, amssymb}  % Some math symbols
\usepackage{enumerate}

\usepackage{graphicx}
\usepackage{hyperref}
\usepackage[all]{xy}
\usepackage{wrapfig}
\usepackage{fancyvrb}
\usepackage[T1]{fontenc}
\usepackage{listings}
\usepackage[shortlabels]{enumitem}

\usepackage{centernot}
\usepackage{mathtools}
\DeclarePairedDelimiter{\ceil}{\lceil}{\rceil}
\DeclarePairedDelimiter{\floor}{\lfloor}{\rfloor}
\DeclarePairedDelimiter{\card}{\vert}{\vert}


\setlength{\parindent}{0pt}
\setlength{\parskip}{5pt plus 1pt}
\pagestyle{empty}

\def\indented#1{\list{}{}\item[]}
\let\indented=\endlist

\newcounter{questionCounter}
\newcounter{partCounter}[questionCounter]

\newenvironment{namedquestion}[1][\arabic{questionCounter}]{%
    \addtocounter{questionCounter}{1}%
    \setcounter{partCounter}{0}%
    \vspace{.2in}%
        \noindent{\bf #1}%
    \vspace{0.3em} \hrule \vspace{.1in}%
}{}

\newenvironment{numedquestion}[0]{%
	\stepcounter{questionCounter}%
    \vspace{.2in}%
        \ifx\writtensection\undefined
        \noindent{\bf \questiontype \; \arabic{questionCounter}. }%
        \else
          \if\writtensection0
          \noindent{\bf \questiontype \; \arabic{questionCounter}. }%
          \else
          \noindent{\bf \questiontype \; \writtensection.\arabic{questionCounter} }%
        \fi
    \vspace{0.3em} \hrule \vspace{.1in}%
}{}

\newenvironment{alphaparts}[0]{%
  \begin{enumerate}[label=\textbf{(\alph*)}]
}{\end{enumerate}}

\newenvironment{arabicparts}[0]{%
  \begin{enumerate}[label=\textbf{\arabic{questionCounter}.\arabic*})]
}{\end{enumerate}}

\newenvironment{questionpart}[0]{%
  \item
}{}

\newcommand{\answerbox}[1]{
\begin{framed}
\vspace{#1}
\end{framed}}

\pagestyle{head}

\headrule
\header{\textbf{\myclass\ \mylecture\mysection}}%
{\textbf{\myname\ (\myUCO)}}%
{\textbf{\myhwtype\ \myhwnum}}

\begin{document}
\thispagestyle{plain}
\begin{center}
  {\Large \myclass{} \myhwtype{} \myhwnum} \\
  \myname{} (\myUCO{}) \\
  \today
\end{center}


%Here you can enter answers to homework questions

\begin{numedquestion}
\begin{enumerate}[a]
    \item We can first observe the following:\\
    \begin{enumerate}
        \item set A = A $\cap$ B + A $\backslash$ B. 
        \item set B = A $\cap$ B + B $\backslash$ A.
        \item A $\cup$ B = A $\cap$ B + A $\backslash$ B + B $\backslash$ A.
    \end{enumerate}
    If we define the function f'(S,T) to be the number of edges from S to T and S and T are disjoint, we can see that 
    \begin{enumerate}
        \item f(A) = f'(A, V $\backslash$ A)
        \item f(B) = f'(B,V $\backslash$ B)
        \item f(A $\cap$ B) = f'(A $\cap$ B, V $\backslash$ (A $\cap$ B))
        \item f(A $\cup$ B) = f'(A $\cup$ B, V $\backslash$ (A $\cup$ B))
    \end{enumerate}
    We break down each of the above to prove the function f is submodular.\\
    
    f'(A, V $\backslash$ A) = f'(A $\cap$ B, V $\backslash$ (A $\cup$ B) ) + f'(A $\cap$ B, B $\backslash$ A) + f'(A $\backslash$ B, B $\backslash$ A) + f'(A $\backslash$ B, V $\backslash$ (A $\cup$ B) ) \\
    \\
    f'(B, V $\backslash$ B) = f'(A $\cap$ B, V $\backslash$ (A $\cup$ B) ) + f'(A $\cap$ B, A $\backslash$ B) + f'(B $\backslash$ A, A $\backslash$ B) + f'(B $\backslash$ A, V $\backslash$ (A $\cup$ B) )\\
    \\
    f'(A $\cup$ B, V $\backslash$ (A $\cup$ B)) = f'(A $\cap$ B, V $\backslash$ (A $\cup$ B)) + f'(A $\backslash$ B,V $\backslash$ (A $\cup$ B)) + f'(B $\backslash$ A, V $\backslash$ (A $\cup$ B))\\
    \\
    f'(A $\cap$ B, V $\backslash$ (A $\cap$ B)) = f'(A $\cap$ B, V $\backslash$ (A $\cup$ B)) + f'(A $\cap$ B, A $\backslash$ B) + f'(A $\cap$ B, B $\backslash$ A)\\
    \\
    
    From the above, we can see that f(A) + f(B) - f(A $\cup$ B) - f(A $\cap$ B) = f'(A, V $\backslash$ A) + f'(B,V $\backslash$ B) - f'(A $\cap$ B, V $\backslash$ (A $\cap$ B)) - f'(A $\cup$ B, V $\backslash$ (A $\cup$ B)) = f'(A $\backslash$ B, B $\backslash$ A) + f'(B $\backslash$ A, A $\backslash$ B) $\geq$ 0.
    
    Hence, f(A) + f(B) $\geq$ f(A $\cup$ B) + f(A $\cap$ B) and f is submodular
    \pagebreak
    \item We prove both directions 
    \begin{enumerate}
        \item if G is Eulerian, f is symmetric: \\
        if G is Eulerian, we know every vertex has the same number of outgoing edges as incoming edges. We prove f is symmetric using induction on the size of S \\
        \\
        \textbf{Base case:} when |S| = 1, meaning there is only 1 vertex in the set, f(S) is the number of outgoing edges and f(V $\backslash$ S) is the number of edges going from V $\backslash$ S to S, which is also the number of incoming edges of S. Since G is Eulerian, f(S) = f(V $\backslash$ S)\\
        \\
        \textbf{Induction Hypothesis:} when |S| = k $\leq$ |V| - 1, f(S) = f(V $\backslash$ S)\\
        \\
        \textbf{Inductive Step:} when |S| = k + 1, we added a new vertex v to the previous set S'. We define the function f'(S,T) to be the number of edges from S to T and S and T are disjoint. Hence f(S) = f(S') + f'(v,V $\backslash$ S) - f'(S',v) and f(V $\backslash$ S) = f(V $\backslash$ S') + f'(V $\backslash$ S,v) - f'(v,S'). Since f'(v,V $\backslash$ S) - f'(S',v) = f'(V $\backslash$ S,v) - f'(v,S'), we can conclude f(S) = f((V $\backslash$ S). Hence, f is symmetric 
        
        \item if f is symmetric, G is Eulerian 
        If f is symmetric, we know f(S) = f(V $\backslash$ S). If |S| = 1, S contains only one vertex v and f(S) is the number of outgoing edges of v and f(V $\backslash$ S) is the number of incoming edges to v. Hence, the number of outgoing edges is the same as the number of outgoing edges for a single vertex. This is true for all vertices, hence G is Eulerian   
        
    \end{enumerate}
    \item From question 2, we know if G is Eulerian, we have f(A) = f(V $\backslash$ A) and f(B) = f(V $\backslash$ B). Since f(V $\backslash$ A) $\geq$ f(B $\backslash$ A) and f(V $\backslash$ B) $\geq$ f(A $\backslash$ B), we know f(A) + f(B) $\geq$ f(B $\backslash$ A) + f(A $\backslash$ B)     
\end{enumerate} 
  
\end{numedquestion}

\pagebreak
\begin{numedquestion}

Suppose (S, T) is a valid minimum (s,t)-cut of G, where $s \in S$ ans $t \in T$. Then u either belongs to S or T. Suppose $u \in S$. In this case, (S, T) is also a minimum cut for (u, t). Therefore $\lambda$(s, t) = $\lambda$(u, t). Suppose $\lambda$(s, u) > $\lambda$(u, t), then the goal becomes to prove $\lambda$(s, t) >= $\lambda$(u, t), which is true because $\lambda$(s, t) = $\lambda$(u, t). Suppose $\lambda$(s, u) $\leq$ $\lambda$(u, t), then the goal becomes to prove $\lambda$(s, t) >= $\lambda$(s, u). Since $\lambda$(s, t) = $\lambda$(u, t), the goal is equivalent to proving $\lambda$(u, t) >= $\lambda$(s, u), which is the hypothesis we assumed for this case. Therefore, $\lambda$(s, t) >= min { $\lambda$(s, u), $\lambda$(u, t) }. The same proof logic can be used to prove the $u \in T$ case.


     
\end{numedquestion}

% if you do not solve some of the questions use this command to increment counter
%\setcounter{questionCounter}{4}
%\begin{numedquestion}
%  Questions 2 and 3 were not solved, this is an answer to question 5.
%\end{numedquestion}


% if questions have subparts, use this command
%\pagebreak
%\begin{numedquestion}
%  Use the alphaparts environment to for letters.
%  \begin{alphaparts}
%    \item Part a
%    \item Part b
%    \item Part c
%  \end{alphaparts}
%\end{numedquestion}


%\begin{numedquestion}
%  Using the \texttt{description} environment is a great way to typeset induction proofs!
%  \begin{description}
%    \item[Base Case:]
%      Here I have my base case.
%    \item[Induction Hypothesis:]
%      Assume things to make proof work. 
%    \item[Induction Step:]
%      Prove all the things.
%  \end{description}

%  Therefore, we have proven the claim by induction on in the \texttt{description} environment.
%\end{numedquestion}



\end{document}
