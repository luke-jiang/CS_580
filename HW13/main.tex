\documentclass[11pt]{exam}
\newcommand{\myname}{Sihao Yin, Yuxuan Jiang} %Write your name in here

\newcommand{\myUCO}{0028234022, 0028440468} %write your UCO in here

\newcommand{\myhwtype}{Homework}
\newcommand{\myhwnum}{13} %Homework set number
\newcommand{\myclass}{CS580}
\newcommand{\mylecture}{}
\newcommand{\mysection}{}
\usepackage{listings}
% Prefix for numedquestion's
\newcommand{\questiontype}{Question}

% Use this if your "written" questions are all under one section
% For example, if the homework handout has Section 5: Written Questions
% and all questions are 5.1, 5.2, 5.3, etc. set this to 5
% Use for 0 no prefix. Redefine as needed per-question.
\newcommand{\writtensection}{0}

\usepackage{amsmath, amsfonts, amsthm, amssymb}  % Some math symbols
\usepackage{enumerate}

\usepackage{graphicx}
\usepackage{hyperref}
\usepackage[all]{xy}
\usepackage{wrapfig}
\usepackage{fancyvrb}
\usepackage[T1]{fontenc}
\usepackage{listings}
\usepackage[shortlabels]{enumitem}

\usepackage{centernot}
\usepackage{mathtools}
\DeclarePairedDelimiter{\ceil}{\lceil}{\rceil}
\DeclarePairedDelimiter{\floor}{\lfloor}{\rfloor}
\DeclarePairedDelimiter{\card}{\vert}{\vert}


\setlength{\parindent}{0pt}
\setlength{\parskip}{5pt plus 1pt}
\pagestyle{empty}

\def\indented#1{\list{}{}\item[]}
\let\indented=\endlist

\newcounter{questionCounter}
\newcounter{partCounter}[questionCounter]

\newenvironment{namedquestion}[1][\arabic{questionCounter}]{%
    \addtocounter{questionCounter}{1}%
    \setcounter{partCounter}{0}%
    \vspace{.2in}%
        \noindent{\bf #1}%
    \vspace{0.3em} \hrule \vspace{.1in}%
}{}

\newenvironment{numedquestion}[0]{%
	\stepcounter{questionCounter}%
    \vspace{.2in}%
        \ifx\writtensection\undefined
        \noindent{\bf \questiontype \; \arabic{questionCounter}. }%
        \else
          \if\writtensection0
          \noindent{\bf \questiontype \; \arabic{questionCounter}. }%
          \else
          \noindent{\bf \questiontype \; \writtensection.\arabic{questionCounter} }%
        \fi
    \vspace{0.3em} \hrule \vspace{.1in}%
}{}

\newenvironment{alphaparts}[0]{%
  \begin{enumerate}[label=\textbf{(\alph*)}]
}{\end{enumerate}}

\newenvironment{arabicparts}[0]{%
  \begin{enumerate}[label=\textbf{\arabic{questionCounter}.\arabic*})]
}{\end{enumerate}}

\newenvironment{questionpart}[0]{%
  \item
}{}

\newcommand{\answerbox}[1]{
\begin{framed}
\vspace{#1}
\end{framed}}

\pagestyle{head}

\headrule
\header{\textbf{\myclass\ \mylecture\mysection}}%
{\textbf{\myname\ (\myUCO)}}%
{\textbf{\myhwtype\ \myhwnum}}

\begin{document}
\thispagestyle{plain}
\begin{center}
  {\Large \myclass{} \myhwtype{} \myhwnum} \\
  \myname{} (\myUCO{}) \\
  \today
\end{center}


%Here you can enter answers to homework questions

\begin{numedquestion}
According to the lecture, Prob(some bin has $\le$ k balls) $\le$ m * ${(\frac{n}{m})}^k$ * $\frac{1}{k!}$. We want to make this probability less than $\frac{1}{n(1+2l)!}$ so that with this probability, k $\ge$ 1+2l
\\
\\
Step1: we have  m * ${(\frac{n}{m})}^k$ * $\frac{1}{k!}$ $\le$ $\frac{1}{n(1+2l)!}$
\\
Step2: Hence, we have ${(\frac{m}{n})}^k$ * k! $\ge$ mn(1+2l)!
\\
Step3: Since m $\ge$ ${n}^{1+\frac{1}{l}}$, we have ${n}^{\frac{k}{l}}$ * k! $\ge$ ${n}^{2+\frac{1}{l}}$ * (1+2l)!
\\
Step4: The above can be simplified to ${n}^{\frac{k}{l}}$ * k! $\ge$ ${n}^{\frac{2l+1}{l}}$ * (1+2l)!
\\
Hence, k $\ge$ (1+2l)
\\

Hence, we proved that the max load is at most 1 + 2l with probability greater than 1 - $\frac{1}{n(1+2l)!}$
\end{numedquestion}

\pagebreak
\begin{numedquestion}
First, we show k! $\ge$ ${(\frac{k}{2})}^{\frac{k}{2}}$\\
Since k! = k(k-1)(k-2)....($\frac{k}{2}$)($\frac{k}{2}$-1)....(1), we know k! $\ge$ k(k-1)(k-2)....($\frac{k}{2}$). Since there are $\frac{k}{2}$ terms in k(k-1)(k-2)....($\frac{k}{2}$) and the terms are decreasing from k all the way to $\frac{k}{2}$, we have k! $\ge$ ${\frac{k}{2}} ^\frac{k}{2} $  \\


If m = n, Prob(some bin has $\le$ k balls) $\le$ $\frac{1}{k!}$ $\le$ $\frac{1}{k^\frac{k}{2}}$, using the property we just proved.\\
Step1: Choose $\alpha = 8$, so we have $k = \frac{8log(n)}{log(log(n))}$. and $\frac{k}{2} = \frac{4log(n)}{log(log(n))}$.\\
Step 2: Using L'Hôpital's rule, we can prove that $log(n) \ge (log(log(n)))^2$: $\lim_{n \to +\infty} \frac{log(n)}{(log(log(n))^2} = \lim_{n \to +\infty} \frac{log(n)}{2(log(log(n))} = +\infty$. Therefore, $64(log(n))^2 \ge log(n) * (log(log(n)) ^ 2$, which impiles $\frac{64(log(n))^2}{(log(log(n)))^2} \ge log(n)$.\\ Step3: Taking square root for both sides, we get $k \ge \sqrt{log(n)}$. \\
Step4: Therefore, $k^\frac{k}{2} \ge (\sqrt{log(n)})^\frac{4log(n)}{log(log(n))} = log(n)^\frac{2log(n)}{log(log(n))} = 2^{2log(n)}$, using the property that $x^{\frac{1}{log(x)}} = 2$ for all x. Choose the base of the logarithms as 2. Then, $2^{2log(n)} = n^2$. \\
Therefore, Prob(some bin has $\ge$ k balls) $\le 1 - n * \frac{1}{n^2} = \frac{1}{n}$.

\end{numedquestion}


% if you do not solve some of the questions use this command to increment counter
%\setcounter{questionCounter}{4}
%\begin{numedquestion}
%  Questions 2 and 3 were not solved, this is an answer to question 5.
%\end{numedquestion}


% if questions have subparts, use this command
%\pagebreak
%\begin{numedquestion}
%  Use the alphaparts environment to for letters.
%  \begin{alphaparts}
%    \item Part a
%    \item Part b
%    \item Part c
%  \end{alphaparts}
%\end{numedquestion}


%\begin{numedquestion}
%  Using the \texttt{description} environment is a great way to typeset induction proofs!
%  \begin{description}
%    \item[Base Case:]
%      Here I have my base case.
%    \item[Induction Hypothesis:]
%      Assume things to make proof work. 
%    \item[Induction Step:]
%      Prove all the things.
%  \end{description}

%  Therefore, we have proven the claim by induction on in the \texttt{description} environment.
%\end{numedquestion}



\end{document}
